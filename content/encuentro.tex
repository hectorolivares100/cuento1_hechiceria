% \lettrine{}{} command typesets a drop cap in the document

% second brace set formats text placed inside as small caps
% (same thing as doing \lettrine{L}\textsc{orem ipsum ...}

\chapter{Encuentro}
\lettrine{E}{se día, igual que todos los domingos,} Melchor se levantó temprano para llegar al ensayo del coro previo a la misa y comenzó su rutina diaria de aseo. No le molestaba levantarse temprano, y de hecho lo hacía sentir bien.
% Desde cuando va, edad del personaje, a qué hora es, que es temprano.
Llevaba yendo a tocar en el coro hacía ya cinco años, desde que tenía doce.

% Describe la rutina, lo importante que es el arreglo y la apariencia.
Se encaminó hacia el baño, que quedaba al lado opuesto del pasillo respecto a su habitación, y tomó una ducha. Salió del baño vestido sólo con pantalón y con el torso desnudo, simplemente porque le gustaba regresar a su habitación y darse una mirada su espejo de cuerpo entero
en cierto momento de su rutina de acicalamiento.
Si alguien lo hubiera cuestionado en el trayecto, habría dicho que aún no decidía qué camisa ponerse, por lo que elegiría entre las que colgaban planchadas en su armario. Nunca admitiría ese pequeño momento de vanidad, que sin embargo ya se había vuelto parte de su rutina diaria. Pero nunca nadie en la casa lo había
cuestionado respecto a eso, y de hecho rara vez lo cuestionaban respecto a algo.
Así que cruzó el pasillo, entró en su habitación y cerró la puerta.

Primero se peinó con ayuda de los dedos y un poco de gel, o más bien bastante.
Aunque los habitantes más viejos de su casa vieran sólo un peinado de pelos parados,
la verdad es que estaba realizado con sumo cuidado y tomaba su tiempo.
La parte central de la cabeza tenía los cabellos apuntando hacia arriba, mientras que los
laterales, de cabello mucho más corto debían quedar más aplacados.
Al mismo tiempo, la elevación central debía librar el contorno del sombrero, para que éste no
se lo desordenara, lo cual también requería cortes de cabello frecuentes.

Una vez satisfecho, llegó el momento de alejarse unos pasos y verse en el espejo.
Así como llevaba años yendo a tocar en el coro de la iglesia, también llevaba ya algunos años entrenando,
y la verdad se notaban los resultados.
Sus brazos y hombros se notaban fuertes, y en su abdomen se notaba el contorno de los típicos cuadritos.
Después de ensayar unas dos o tres miradas y poses, se acabó de vestir con una camisa de manga larga a cuadros, no sin antes rociarse una buena dosis de desodorante en aerosol.
Se arremangó la camisa a mitad del antebrazo, calzó las botas, se puso el sombrero y se colgó la
guitarra dentro de su funda. Estaba listo para salir.


% El camino, el frío, nubes, cerros

% El ensayo (más bien preparación). Ayuda a afinar, repaso de los requintos y del salmo. Describe la actitud del protagonista y los demás miembros del coro. Todos lo esperan y lo respetan. Él se sabe el centro de atención y lo disfruta, pero hace gala de la aparente virtud de la modestia, porque sabe que es lo correcto, no sin dejar de lucirla.

% La misa transcurre normalmente, no abundar en detalles.

% Después de la misa, llega la chica (Jennyfer) a unirse al coro

% El ensayo
% Describe a los personajes: el organista, los otros miembros del coro, chicos y chicas. (¿tal vez en una ronda de presentaciones?)

% El organista le sugiere a la chica aprender guitarra de Melchor.

% Jennyfer un poco tímida (aunque se ve que no gusta mucho de esas músicas, prefiere otro estilo). Al darle la oportunidad de hablar e integrarse, involuntariamente deja en ridículo a Melchor (por alguna cuestión de ignorancia ¿o quizás al demostrar sin querer que es mejor que él?). Este se desquita y cada uno se gana el desprecio del otro.

% Aún no sé en qué acaba.

