% \lettrine{}{} command typesets a drop cap in the document

% second brace set formats text placed inside as small caps
% (same thing as doing \lettrine{L}\textsc{orem ipsum ...}

\chapter{Encuentro}
\lettrine{E}{se día, igual que todos los domingos,} Melchor se levantó temprano para llegar al ensayo del coro previo a la misa. No le molestaba levantarse temprano, y de hecho lo hacía sentir bien.
% Desde cuando va, edad del personaje, a qué hora es, que es temprano.
Llevaba yendo a tocar en el coro hacía ya cinco años, desde que tenía doce años.


% Describe la rutina, lo importante que es el arreglo y la apariencia.

% El camino, el frío, nubes, cerros

% El ensayo (más bien preparación). Ayuda a afinar, repaso de los requintos y del salmo. Describe la actitud del protagonista y los demás miembros del coro. Todos lo esperan y lo respetan. Él se sabe el centro de atención y lo disfruta, pero hace gala de la aparente virtud de la modestia, porque sabe que es lo correcto, no sin dejar de lucirla.

% La misa transcurre normalmente, no abundar en detalles.

% Después de la misa, llega la chica (Jennyfer) a unirse al coro

% El ensayo
% Describe a los personajes: el organista, los otros miembros del coro, chicos y chicas. (¿tal vez en una ronda de presentaciones?)

% El organista le sugiere a la chica aprender guitarra de Melchor.

% Jennyfer un poco tímida (aunque se ve que no gusta mucho de esas músicas, prefiere otro estilo). Al darle la oportunidad de hablar e integrarse, involuntariamente deja en ridículo a Melchor (por alguna cuestión de ignorancia ¿o quizás al demostrar sin querer que es mejor que él?). Este se desquita y cada uno se gana el desprecio del otro.

% Aún no sé en qué acaba.

